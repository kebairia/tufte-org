% Created 2021-10-26 Tue 09:30
% Intended LaTeX compiler: lualatex
\documentclass{tufte-org}
\usepackage{graphicx}
\usepackage{grffile}
\usepackage{longtable}
\usepackage{wrapfig}
\usepackage{rotating}
\usepackage[normalem]{ulem}
\usepackage{amsmath}
\usepackage{textcomp}
\usepackage{amssymb}
\usepackage{capt-of}
\usepackage{hyperref}
\usepackage{minted}
\usepackage{xcolor}

\definecolor{hd-red}   {RGB}{197, 13, 41}
\definecolor{hd-brown} {RGB}{ 90, 15, 20}
\definecolor{hd-beige} {RGB}{245, 240, 234}
\definecolor{hd-yellow}{RGB}{255, 191, 0}
\definecolor{hd-blue}  {RGB}{70, 129, 180}
\renewcommand{\floatpagefraction}{.8}%
\DeclareSIUnit\px{px}
\RequirePackage[%
colorlinks = true,
citecolor  = hd-red,
linkcolor  = hd-red,
urlcolor   = hd-red,
]{hyperref}
\RequirePackage{bookmark}
\usepackage{bookmark}
\bookmarksetup{depth=2}
\usepackage{bbding}
\def\signed #1{{\leavevmode\unskip\nobreak\hfil\penalty50\hskip1em
\hbox{}\nobreak\hfill #1%
\parfillskip=0pt \finalhyphendemerits=0 \endgraf}}
\newsavebox\mybox
\newenvironment{aquote}[1]
{\savebox\mybox{#1}\begin{quote}\openautoquote\hspace*{-.7ex}}
{\unskip\closeautoquote\vspace*{1mm}\signed{\usebox\mybox}\end{quote}}
\usepackage{hyphenat}
\hyphenation{deoxy-hemo-glo-bin}
\hyphenation{Sie-mens}
\hyphenation{multi-band multi-echo}
\usepackage[level]{datetime}
\setstretch{1.25}
\setparsizes{0em}{0.1\baselineskip plus .1\baselineskip}{1em plus 1fil}
\renewcommand{\th}{\textsuperscript{\textup{th}}\xspace}
\usepackage{tikz}
\usepackage{pgfplots}
\pgfplotsset{compat=1.12}

% Permits accessing the smallest and largest x value of a plot
\makeatletter
\newcommand{\pgfplotsxmin}{\pgfplots@xmin}
\newcommand{\pgfplotsxmax}{\pgfplots@xmax}
\makeatother

% Permits accessing the smallest and largest y value of a plot
\makeatletter
\newcommand{\pgfplotsymin}{\pgfplots@ymin}
\newcommand{\pgfplotsymax}{\pgfplots@ymax}
\makeatother
\renewcommand{\floatpagefraction}{.8}%
\usepackage{metalogo}
\usepackage{etoolbox}

\usepackage[binary-units=true]{siunitx}
\DeclareSIUnit\px{px}

\sisetup{%
detect-all           = true,
detect-family        = true,
detect-mode          = true,
detect-shape         = true,
detect-weight        = true,
detect-inline-weight = math,
}
\usepackage{indentfirst}
\setlength{\parindent}{0.2cm} % Default is 15pt.
\usepackage[%
autocite     = plain,
backend      = biber,
doi          = true,
url          = true,
giveninits   = true,
hyperref     = true,
maxbibnames  = 99,
maxcitenames = 99,
sortcites    = true,
style        = numeric,
]{biblatex}

%%%%%%%%%%%%%%%%%%%%%%%%%%%%%%%%%%%%%%%%%%%%%%%%%%%%%%%%%%%%%%%%%%%%%%%%
% Some adjustments to make the bibliography more clean
%%%%%%%%%%%%%%%%%%%%%%%%%%%%%%%%%%%%%%%%%%%%%%%%%%%%%%%%%%%%%%%%%%%%%%%%
%
% The subsequent commands do the following:
%  - Removing the month field from the bibliography
%  - Fixing the Oxford commma
%  - Suppress the "in" for journal articles
%  - Remove the parentheses of the year in an article
%  - Delimit volume and issue of an article by a colon ":" instead of
%    a dot ""
%  - Use commas to separate the location of publishers from their name
%  - Remove the abbreviation for technical reports
%  - Display the label of bibliographic entries without brackets in the
%    bibliography
%  - Ensure that DOIs are followed by a non-breakable space
%  - Use hair spaces between initials of authors
%  - Make the font size of citations smaller
%  - Fixing ordinal numbers (1st, 2nd, 3rd, and so) on by using
%    superscripts

% Remove the month field from the bibliography. It does not serve a good
% purpose, I guess. And often, it cannot be used because the journals
% have some crazy issue policies.
\AtEveryBibitem{\clearfield{month}}
\AtEveryCitekey{\clearfield{month}}

% Fixing the Oxford comma. Not sure whether this is the proper solution.
% More information is available under [1] and [2].
%
% [1] http://tex.stackexchange.com/questions/97712/biblatex-apa-style-is-missing-a-comma-in-the-references-why
% [2] http://tex.stackexchange.com/questions/44048/use-et-al-in-biblatex-custom-style
%
\AtBeginBibliography{%
  \renewcommand*{\finalnamedelim}{%
    \ifthenelse{\value{listcount} > 2}{%
      \addcomma
      \addspace
      \bibstring{and}%
    }{%
      \addspace
      \bibstring{and}%
    }
  }
}

% Suppress "in" for journal articles. This is unnecessary in my opinion
% because the journal title is typeset in italics anyway.
\renewbibmacro{in:}{%
  \ifentrytype{article}
  {%
  }%
  % else
  {%
    \printtext{\bibstring{in}\intitlepunct}%
  }%
}

% Remove the parentheses for the year in an article. This removes a lot
% of undesired parentheses in the bibliography, thereby improving the
% readability. Moreover, it makes the look of the bibliography more
% consistent.
\renewbibmacro*{issue+date}{%
  \setunit{\addcomma\space}
    \iffieldundef{issue}
      {\usebibmacro{date}}
      {\printfield{issue}%
       \setunit*{\addspace}%
       \usebibmacro{date}}%
  \newunit}

% Delimit the volume and the number of an article by a colon instead of
% by a dot, which I consider to be more readable.
\renewbibmacro*{volume+number+eid}{%
  \printfield{volume}%
  \setunit*{\addcolon}%
  \printfield{number}%
  \setunit{\addcomma\space}%
  \printfield{eid}%
}

% Do not use a colon for the publisher location. Instead, connect
% publisher, location, and date via commas.
\renewbibmacro*{publisher+location+date}{%
  \printlist{publisher}%
  \setunit*{\addcomma\space}%
  \printlist{location}%
  \setunit*{\addcomma\space}%
  \usebibmacro{date}%
  \newunit%
}

% Ditto for other entry types.
\renewbibmacro*{organization+location+date}{%
  \printlist{location}%
  \setunit*{\addcomma\space}%
  \printlist{organization}%
  \setunit*{\addcomma\space}%
  \usebibmacro{date}%
  \newunit%
}

% Display the label of a bibliographic entry in bare style, without any
% brackets. I like this more than the default.
%
% Note that this is *really* the proper and official way of doing this.
\DeclareFieldFormat{labelnumberwidth}{#1\adddot}

% Ensure that DOIs are followed by a non-breakable space.
\DeclareFieldFormat{doi}{%
  \mkbibacro{DOI}\addcolon\addnbspace
    \ifhyperref
      {\href{http://dx.doi.org/#1}{\nolinkurl{#1}}}
      %
      {\nolinkurl{#1}}
}

% Use proper hair spaces between initials as suggested by Bringhurst and
% others.
\renewcommand*\bibinitdelim {\addnbthinspace}
\renewcommand*\bibnamedelima{\addnbthinspace}
\renewcommand*\bibnamedelimb{\addnbthinspace}
\renewcommand*\bibnamedelimi{\addnbthinspace}

% Make the font size of citations smaller. Depending on your selected
% font, you might not need this.
\renewcommand*{\citesetup}{%
  \biburlsetup
  \small
}

\DeclareLanguageMapping{english}{english-mimosis}

\AtEveryBibitem{%
\clearfield{urlyear}
\clearfield{urlmonth}
\clearfield{note}
\clearfield{issn} % Remove issn
\ifentrytype{online}{}{% Remove url except for @online
\clearfield{url}
}
}
\setlength\bibitemsep{1.1\itemsep}
\renewcommand*{\bibfont}{\footnotesize}
\makeatletter
\renewcommand{\partformat}{\huge\partname~\color{hd-red}\thepart\autodot}
\renewcommand*{\chapterformat}{  \mbox{\chapappifchapterprefix{\nobreakspace}{\color{hd-red}\fontsize{50}{55}\selectfont\thechapter}\autodot\enskip}}
\renewcommand\@seccntformat[1]{\color{hd-red} {\csname the#1\endcsname}\hspace{0.3em}}
\makeatother
\numberwithin{equation}{chapter}
\usepackage{listings}
\usepackage{minted}
\newminted{python}{frame=single,framerule=2pt}
\setminted{autogobble=true,fontsize=\small,baselinestretch=0.8,breaklines,linenos}
\setminted[python]{python3=true,tabsize=4}
\usemintedstyle{trac}
\lstset{abovecaptionskip=0}
\numberwithin{listing}{chapter}
\renewcommand{\listingscaption}{Code Snippet}
\usepackage{scrhack}
\bibliography{~/org/bib/refs}
\pagenumbering{arabic}
\author{Kebairia Zakaria}
\date{\today}
\title{An introduction}
\hypersetup{
 pdfauthor={Kebairia Zakaria},
 pdftitle={An introduction},
 pdfkeywords={},
 pdfsubject={},
 pdfcreator={Emacs 27.2 (Org mode 9.4.6)}, 
 pdflang={English}}
\begin{document}

\setcounter{tocdepth}{1}
 
\frontmatter
\glsenablehyper
\tableofcontents
\mainmatter


\section{Introduction}
\label{sec:org1451498}
This \textbf{package} contains a minimal, modern template for writing your \autoref{fig:myfig1}
thesis. While originally meant to be used for a Ph.D. thesis, you can
equally well use it for your honour thesis, bachelor thesis, and so
on---some adjustments may be necessary, though.
these are citation \cite{Laramee11,Laramee10} and \cite{Edelsbrunner02}, \cite{Edelsbrunner10}
so be \autocite{Tufte01}
\subsection{Why?}
\label{sec:org928e55e}
I was not satisfied with the available templates for and wanted
to heed the style advice given by people such as Robert Bringhurst \cite{Bringhurst12} or Edward R.
Tufte \cite{Tufte90,Tufte01} . While there \emph{are} some packages out 
there that attempt to emulate these styles, I found them to be either
too bloated, too playful, or too constraining. This template attempts to
produce a beautiful look without having to resort to any sort of hacks.
I hope you like it \cite{nikouei19:_i_safe}.

\subsection{How?}
\label{sec:org83a9687}
The package tries to be easy to use. If you are satisfied with the
default settings, just add \texttt{\textbackslash{}documentclass\{mimosis\}} at the beginning of your document.
This is sufficient to use the class.
It is possible to build your document using either  or. I personally prefer one of the latter two because they make
it easier to select proper fonts.
\subsection{Features}
\label{sec:org46b5bc3}
\subsubsection{Tables}
\label{sec:org3af9c3c}

\begin{table}[htbp]
\caption{\label{table1}Testing table}
\centering
\begin{tabular}{ll}
\toprule
\textbf{Package} & \textbf{Purpose}\\
\midrule
math & for math and other\\
matplotlib & for ploting with python\\
another test & another purpose\\
\bottomrule
\end{tabular}
\end{table}
The template automatically imports numerous convenience packages that
aid in your typesetting process.
most important ones. Let's briefly discuss some examples below. Please
refer to the source code for more demonstrations.
\subsubsection{images}
\label{sec:orgcd41f70}
\subsubsection{Equations}
\label{sec:org4833222}
Define new mathematical operators using \verb|\DeclareMathOperator|.
Some operators are already pre-defined by the template, such as the
distance between two objects. Please see the template for some examples. 
\%
Moreover, this template contains a correct differential operator. Use \verb|\diff| to typeset the differential of integrals:
\begin{equation}\label{eq:equation1}
  f(u) := \int_{v \in \domain}\dist(u,v)\diff{v}
\end{equation}
You can see that, as a courtesy towards most mathematicians, this
template gives you the possibility to refer to the real numbers 
and the domain \textasciitilde{}\(\domain\) of some function. Take a look at the source for
more examples. By the way, the template comes with spacing fixes for the
automated placement of brackets.
and to the equation \eqref{eq:equation1}

\section[Configuration]{Chapter 2}
\label{sec:org5863e35}
\section[Conclusion]{Chapter 3}
\label{sec:orgc62cce3}
\appendix
\section{Emacs and Org mode}
\label{sec:orgde349be}
this document is made using \gls{latex} and Emacs

% This ensures that the subsequent sections are being included as root
% items in the bookmark structure of your PDF reader.
\bookmarksetup{startatroot}
\backmatter
\begingroup
    \let\clearpage\relax
    \glsaddall
    \printglossary[type=\acronymtype]
    \newpage
    \printglossary
\endgroup
\printindex

\printbibliography
%% \bibliographystyle{unsrt}
%% \bibliography{./lib/refs}
\end{document}