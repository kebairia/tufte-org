% Created 2021-10-26 Tue 17:47
% Intended LaTeX compiler: pdflatex
\documentclass{tufte-org}
\usepackage[utf8]{inputenc}
\usepackage[T1]{fontenc}
\usepackage{graphicx}
\usepackage{grffile}
\usepackage{longtable}
\usepackage{wrapfig}
\usepackage{rotating}
\usepackage[normalem]{ulem}
\usepackage{amsmath}
\usepackage{textcomp}
\usepackage{amssymb}
\usepackage{capt-of}
\usepackage{hyperref}
\usepackage{minted}
\bibliography{~/org/bib/refs}
\pagenumbering{arabic}
\author{Kebairia Zakaria}
\date{\today}
\title{An introduction}
\hypersetup{
 pdfauthor={Kebairia Zakaria},
 pdftitle={An introduction},
 pdfkeywords={},
 pdfsubject={},
 pdfcreator={Emacs 27.2 (Org mode 9.4.6)}, 
 pdflang={English}}
\begin{document}

 
\frontmatter
\glsenablehyper
\tableofcontents
\mainmatter


\section{Introduction}
\label{sec:org2ce1b0d}
This \textbf{package} contains a minimal, modern template for writing your \autoref{fig:myfig1}
thesis. While originally meant to be used for a Ph.D. thesis, you can
equally well use it for your honour thesis, bachelor thesis, and so
on---some adjustments may be necessary, though.
these are citation \cite{Laramee11,Laramee10} and \cite{Edelsbrunner02}, \cite{Edelsbrunner10}
so be \autocite{Tufte01}
\subsection{Why?}
\label{sec:orgefc455f}
I was not satisfied with the available templates for and wanted
to heed the style advice given by people such as Robert Bringhurst \cite{Bringhurst12} or Edward R.
Tufte \cite{Tufte90,Tufte01} . While there \emph{are} some packages out 
there that attempt to emulate these styles, I found them to be either
too bloated, too playful, or too constraining. This template attempts to
produce a beautiful look without having to resort to any sort of hacks.
I hope you like it \cite{nikouei19:_i_safe}.

\subsection{How?}
\label{sec:org49b44f6}
The package tries to be easy to use. If you are satisfied with the
default settings, just add \texttt{\textbackslash{}documentclass\{mimosis\}} at the beginning of your document.
This is sufficient to use the class.
It is possible to build your document using either  or. I personally prefer one of the latter two because they make
it easier to select proper fonts.
\subsection{Features}
\label{sec:org04db2b3}
\subsubsection{Tables}
\label{sec:org60e449b}

\begin{table}[htbp]
\caption{\label{table1}Testing table}
\centering
\begin{tabular}{ll}
\toprule
\textbf{Package} & \textbf{Purpose}\\
\midrule
math & for math and other\\
matplotlib & for ploting with python\\
another test & another purpose\\
\bottomrule
\end{tabular}
\end{table}
The template automatically imports numerous convenience packages that
aid in your typesetting process.
most important ones. Let's briefly discuss some examples below. Please
refer to the source code for more demonstrations.
\subsubsection{images}
\label{sec:org3a647b7}
\subsubsection{Equations}
\label{sec:orgcf84e76}
Define new mathematical operators using \verb|\DeclareMathOperator|.
Some operators are already pre-defined by the template, such as the
distance between two objects. Please see the template for some examples. 
\%
Moreover, this template contains a correct differential operator. Use \verb|\diff| to typeset the differential of integrals:
\begin{equation}\label{eq:equation1}
  f(u) := \int_{v \in \domain}\dist(u,v)\diff{v}
\end{equation}
You can see that, as a courtesy towards most mathematicians, this
template gives you the possibility to refer to the real numbers 
and the domain \textasciitilde{}\(\domain\) of some function. Take a look at the source for
more examples. By the way, the template comes with spacing fixes for the
automated placement of brackets.
and to the equation \eqref{eq:equation1}

\section[Configuration]{Chapter 2}
\label{sec:orgaa8841b}
\section[Conclusion]{Chapter 3}
\label{sec:org87ff922}
\appendix
\section{Emacs and Org mode}
\label{sec:orged80eff}
this document is made using \gls{latex} and Emacs

% This ensures that the subsequent sections are being included as root
% items in the bookmark structure of your PDF reader.
\bookmarksetup{startatroot}
\backmatter
\begingroup
    \let\clearpage\relax
    \glsaddall
    \printglossary[type=\acronymtype]
    \newpage
    \printglossary
\endgroup
\printindex

\printbibliography
%% \bibliographystyle{unsrt}
%% \bibliography{./lib/refs}
\end{document}